\documentclass[UTF8]{ctexart}
\usepackage{listings}
\title{数据结构笔记1}
\author{王智浩}
\date{\today}
\begin{document}
  \lstset{numbers=left,
  numberstyle= \tiny,keywordstyle= \color{ blue!70},commentstyle=\color{red!50!green!50!blue!50},
  frame=shadowbox, rulesepcolor= \color{ red!20!green!20!blue!20},
  escapeinside=``}
\maketitle

\newpage
\section {基本概念}
魔方程序
\begin{lstlisting}[language=C]
  #include<stdio.h>
  #include<stdlib.h>
  #define MAX_SIZE 15

  void main(void)
  {
  	static int square[MAX_SIZE][MAX_SIZE];
  	int i, j, row, column;
  	int count;
  	int size;

  	printf("Enter the size of the square:");
  	scanf_s("%d", &size);
  	if (size <1 || size>MAX_SIZE + 1)
  	{
  		fprintf(stderr, "REEOR!Size is out of range\n");
  		exit(1);
  	}
  	if (!(size % 2))
  	{
  		fprintf(stderr, "REEOR!Size is even\n ");
  		exit(1);
  	}
  	for (i = 0; i < size; i++)
  		for (j = 0; j < size; j++)
  			square[i][j] = 0;
  	square[0][(size - 1) / 2] = 1;
  	i = 0;
  	j = (size - 1) / 2;
  	for (count = 2; count <= size*size; count++)
  	{
  		row = (i - 1 < 0) ? (size - 1) : (i - 1);
  		column = (j - 1 < 0) ? (size - 1) : (j - 1);
  		if (square[row][column])
  			i = (++i) % size;
  		else
  		{
  			i = row;
  			j = (j - 1 < 0) ? (size - 1) : --j;
  		}
  		square[i][j] = count;
  	}
  	printf("Magic square of size %d:\n\n", size);
  	for (i = 0; i < size; i++) {
  		for (j = 0; j < size; j++)
  			printf("%5d", square[i][j]);
  		printf("\n");
  	}
  	getchar();
  	printf("\n\n");
  }
\end{lstlisting}

\end{document}
